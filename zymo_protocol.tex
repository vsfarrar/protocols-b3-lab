\documentclass[a4paper,12pt,twoside]{book}
\usepackage[english]{babel}
\usepackage[utf8]{inputenc}
 \usepackage{lipsum}% http://ctan.org/pkg/lipsum
 \usepackage{fancyhdr}% http://ctan.org/pkg/fancyhdr
 \fancyhf{}% Clear all headers/footers
 \renewcommand{\headrulewidth}{0pt}% No header rule
 \renewcommand{\footrulewidth}{0pt}% No footer rule
 \fancyhead[R]{{\em updated 10/29/2019VF}}%
\pagestyle{empty}

\begin{document}
	\thispagestyle{fancy}%
\section*{Zymo Direct-Zol RNA Extraction}
 \subsection*{Purpose:}
To extract high-quality RNA from homogenized tissues.
\subsection*{Reagents:}
\begin{itemize}
	\item Zymo Direct-Zol RNA mini kit (Genesee)
	\item TriSure reagent (BioLine; stored at 4 C)
	\item 100 percent Ethanol
	\item 1.5 mL tubes for storage of extracted RNAs
\end{itemize}
 \subsection*{Instructions:}
\begin{enumerate}
	\item Homogenize tissue in 800 uL TriSure per 25 mg of sample (or less). ({\em See Tissue Homogenziation protocol})
	\subitem Note: Begin thawing DNase I mixture on ice or in fridge after you finish homogenization. DNase I stored in the RNA extraction box in the -20 C freezer. 
	\subitem **After homogenizing, PUT SAMPLES BACK IN - 80 C FREEZER ASAP!!
	\item Add 1 volume (e.g. 800 uL if 800 uL TriSure added) 100 percent EtOH to homogenized sample in TriSure. Vortex about 10 seconds to mix.
	\item Load ~900 uL EtOH-TriSure-sample mixture into column. (The column will be quite full).
	\subitem Spin for 1 minute at 12,000 rpm.  
	\subitem Pour out flow-through into the RNA extraction waste container in the hood. Replace filter in collection tube.
	\item Repeat step 3 until all of the mixture has been loaded onto the column and spun. 
	\item Wash column with 400 uL RNA wash buffer (green cap) before DNA digest step. Centrifuge for 30 sec at 12K rpm. Pour out flow-through.
	\item {\bf DNase treatment:} 
	\subitem Prepare the DNase-buffer mix by mixing the following volumes of DNase buffer and enzyme into a new tube: 
	\subitem DNase buffer = 75 x 1.05 x {\em number of samples}  uL 
	\subitem DNase enzyme =  5 x 1.05 x {\em number of samples} uL 
	\subitem Load 80 uL of this DNase-buffer mix directly onto the column of each sample. Gently touch tip to filter, using a different tip each time. 
	\subitem Let samples stand at room temperature for 15 minutes.  (Now is a good time to put the DNase enzyme back in the - 20 C freezer.)
	\subitem After incubation, centrifuge for 30 sec at 12K rpm. Pour out any flow-through. 
	\item {\bf Prewash:} Add 400 uL pre-wash buffer and centrifuge samples for 1 minute at 12K rpm. Pour out flow-through.
	\item Repeat step 6. 
	\item {\bf Wash column:} Add  700 uL RNA Wash Buffer (green cap) to column directly, touching filter gently and using a different tip for each sample.  Spin for 1 minute at 12K rpm.  Pour out flow-through. 
	\item Spin empty columns for another 1 min at 12K rpm to remove all excess wash buffer. 
	\item{\bf Elution:} Place columns in their final 1.5 mL storage tubes 
	(Final tubes need to be labeled on top and side with sample ID, RNA, and date.) Add 20 uL DNA/RNAse free H2O to each column by touching the filter gently, using a different tip each time.  Centrifuge for 1 minute at 12K rpm. 
	\item Repeat step 10 for a total elution of 40 uL. 
	\item Nanodrop immediately, or store in a labeled rack or box in the -20 C freezer. 
	
	
\end{enumerate}

 \subsection*{Updates:}
Version 1.1: Updated to 6/17/2017 VF  \\*
Version 1.2: Updated to reflect notes 10/29/2019 - VSF

\end{document}