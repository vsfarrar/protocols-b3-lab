\documentclass[a4paper,12pt,twoside]{book}
\usepackage[english]{babel}
\usepackage[utf8]{inputenc}
\usepackage{hyperref}
 \usepackage{lipsum}% http://ctan.org/pkg/lipsum
 \usepackage{fancyhdr}% http://ctan.org/pkg/fancyhdr
 \fancyhf{}% Clear all headers/footers
 \renewcommand{\headrulewidth}{0pt}% No header rule
 \renewcommand{\footrulewidth}{0pt}% No footer rule
 \fancyhead[R]{{\em updated 2/12/2019 VF}}%
\pagestyle{empty}

\begin{document}
	\thispagestyle{fancy}%
\section*{RNA extraction for Lipid-Rich tissues (Zymo kit)}
	
 \subsection*{Purpose:}
To extract high-quality RNA from homogenized tissues, especially lipid-rich tissues such as hypothalamii, brain, thymus, or crop.

\subsection*{Reagents:}
\begin{itemize}
	\item Zymo Direct-Zol RNA mini kit (Genesee)
	\item TriSure reagent (BioLine; stored at 4 C) OR TRIZol reagent (ThermoFisher)
	\item Chloroform (flammables cabinet, hood)
	\item 21 G needles and 1 mL syringes (1 per sample)
	\item 100 percent Ethanol (flammables cabinet)
	\item 1.7 mL tubes for storage of extracted RNAs
\end{itemize}
 \subsection*{Instructions:}
\begin{enumerate}
	\item Homogenize tissue in 750 uL TriSure per {\bf 10 mg} of sample (or less). ({\em See Tissue Homogenization protocol}) 
	\subitem Note: Brain tissue will get foamy during homogenizing. Don't let the foam get out of the tube! 
	\subitem Note: With help, move the centrifuge into the bottom shelf of the fridge to cool it for later spin steps. 
	\item Add 250 uL additional TriSure to homogenized sample in solution. 
	\item {\bf Second homogenization:} Pass entire sample 10 times through clean 21 gauge (26 gauge also worked) needle, attached to a 1 mL syringe. Store used syringes in the "dirty" bag, throw used needles in the sharps waste. 
	\item Let homogenate sit at room temperature for 5 minutes to allow nucleoprotein complexes to dissociate. 
	\item While waiting for above step, {\bf PUT BACK any tissues that are still out in -80 C freezer A.S.A.P!!}
	\item In the hood, pour out only enough chloroform for 200 uL per sample into a large 10 mL falcon tube. (e.g. 2 -3 mL for 10 samples). {\em Chloroform evaporates quickly so we don't want to pour more than is necessary.}
	\item  In the hood, add 200 uL chloroform to each sample. Vortex vigorously for 15 seconds in the hood. 
	\item Let homogenate sit at room temperature for 5 minutes in the hood. 
	\subitem Note: Begin thawing DNase I mixture on ice or in fridge after you finish homogenization. DNase I stored in the RNA extraction box in the -20 C freezer. 
	\item {\bf Phase separation spin:}  In the fridge (4 C), centrifuge samples for 15 minutes at 12,000 rpm. 
	\subitem Note: This is a good time to prepare numbered 2 mL tubes and spin columns/collection tubes for each sample.
	\item  Transfer upper aqueous phase (about 600 uL) to a new collection tube (these are the same 2 mL tubes used in during tissue homogenization). {\em Take care not to get any of the green-turquoise layer into the  tube.}  Pour out other phases into the RNA extraction waste container. 
	\subitem Note how much the aqueous phase is, you will need to add this volume of ethanol in the next step. For example, if your sample only had 500 uL of aqueous phase transferred, only add 500 uL of EtOH in the next step. 
	\item Add 1 volume (e.g. 600  uL in previous step) of 100 percent EtOH to transfered aqueous phase. Mix by pipetting up and down, or vortex at a {\em low} setting. 
	\item[] {\bf The following steps are the same as steps 3 - 12 of the regular Zymo RNA extraction protocol.}
	\item Load 700 uL EtOH-sample mixture into column (= filter + tube from yellow Zymo box)). Spin for 1 minute at 12,000 rpm.  \\
	Pour out flow-through. \\
	{\bf Repeat} until all of the mixture has been loaded onto the column and spun. 
	\item Wash column with 400 uL RNA wash buffer before DNA digest step. Centrifuge for 30 sec at 12K rpm. Pour out flow-through.
	\item {\bf DNase treatment:} 
	\subitem Prepare the DNase-buffer mix by mixing the following volumes of DNase buffer and enzyme into a new tube: 
	\subitem DNase buffer = 75 x 1.05 x {\em number of samples}  uL 
	\subitem DNase enzyme =  5 x 1.05 x {\em number of samples} uL 
	\subitem Load 80 uL of the DNase-buffer mix directly onto the column of each sample. Gently touch tip to filter, use a different tip each time. 
	\subitem Incubate samples at room temperature for 15 minutes.  
	\subsubitem {\em Note: Now is a good time to prepare labeled final 1.7 mL tubes for each sample. Put back DNase in freezer.}
	\subitem After incubation, centrifuge for 30 sec at 12K rpm. Pour out any flow-through. 
	\item {\bf Prewash:} Add 400 uL pre-wash buffer and centrifuge samples for 1 minute at 12K rpm. Pour out flow-through.
	\item Repeat step 15. 
	\item {\bf Wash column:} Add  700 uL RNA Wash Buffer (green cap) to column directly, touching filter gently and using a different tip for each sample.  Spin for 1 minute at 12K rpm.  Pour out flow-through. 
	\item Spin empty columns for another 1 min at 12K rpm to remove all excess wash buffer. 
	\item{\bf Elution:} Place columns in their final 1.5 mL storage tubes 
	(Final tubes need to be labeled on top and side with sample ID, RNA, and date.) 
	\item[] Add 20 uL DNA/RNAse free H2O to each column.  Centrifuge for 1 minute at 12K rpm.  Let sit at room temperature for 1 minute. 
	\item Repeat step 19 for a total elution of 40 uL. 
	\item Nanodrop immediately, or store in a labeled rack or box in the -20 C freezer. 
	\subitem Check one last time to be sure any tissues/dry ice have been put away in the -80 C freezer. 
	
 \subsection*{Source protocol:}	
	Modified from the laboratory of Dr. Christopher Mason "Extract Total RNA from Lipid Tissues" protocol, Cornell University, 2011.
	
	Accessed from: \url{http://physiology.med.cornell.edu/faculty/mason/lab/zumbo/zumbos_documents/DOCUMENTS_files/ZUMBO_rna_isolation_tissues.pdf}

 \subsection*{Updates:}
 Protocol written 7/31/17 by V. Farrar
 
 Version 1.2: Updated to reflect notes by R.Viernes (2018-19)
 
 Version 1.3: Small updates by V. Farrar (10/29/2019)
 
 Version 1.4: Updates for brain nuclei by V.Farrar (2021-02-19)
	
\end{enumerate}

\end{document}